% Awesome CV LaTeX Template
%
% This template has been downloaded from:
% https://github.com/huajh/huajh-awesome-latex-cv
%
% Author:
% Junhao Hua

\sectionTitle{实习经历}{\faBlackTie}
 
\begin{experiences}
  \experiencenew
  {2022.12 -- 2023.03} {软件算法工程师,华为技术有限公司,东莞}
  {
    参与优化模块在复位场景下的性能表现,减少模块消耗的总计算量,包括:
    \begin{itemize}
  		\item 根据火焰图,定位耗时函数并分析原因;
  		\item 根据模块逻辑和数据特点,重新选择容器、优化容器调用方式;
  		\item 无锁DC的编译期冲突检测;
  	\end{itemize}
    分析发现业务代码中大量使用std容器,尤其是使用std::map容器用于维护大量的映射消息,这一部分贡献了大量的耗时。
    针对这种情况,我单独对容器进行了一定的研究:
    \begin{itemize}
  		\item 搭建特定场景多容器性能评估框架;
  		\item 调研std::map的底层逻辑,并进行非侵入式性能优化;
  	\end{itemize}
    同时我还很荣幸受到主管的邀请,在PDU上分享了关于数据结构算法的基础知识和应用,向同事们介绍针对性算法优化对运行效率的提升;
                }
                {C/C++, C++ STL, 算法优化}
\emptySeparator	

  \experiencenew
  {2023.06 -- 2023.09} {技术开发实习生,佳期投资管理有限公司,上海}
  {
    接受量化领域知识训练,独立完成C++ Project,包括:
    \begin{itemize}
  		\item TCP C/S Framework;
  		\item In-memory/Disk KV Storage;
  		\item Backtesting System;
  	\end{itemize}
    其中Backtesting System中包括了以下的模块/功能:
    \begin{itemize}
  		\item Order Book Reconstruction: 复现SZE/SSE交易撮合规则;
  		\item Market Data Loader: 从远端拉取数据并解码,实现本地Disk/Memory Cache;
  		\item Algorithm: 用户自定义策略;
  		\item Latency Simulation: 模拟模块之间的计算/网络延迟;
  	\end{itemize}
    参与了少量Quant Research/Market Study,并以报告的形式提交结果;
                }
                {C/C++, Quantitative Trading}
% \emptySeparator	


  % \emptySeparator
  % \experience
  % {2012年11月} {C/C++ 工程师实习生}{研发部分}{道富信息科技(浙江)有限公司, 杭州}
  % {2012年7月}    {
  % 	\begin{itemize}
  % 		\item  负责\link{http://www.statestreet.com/solutions/by-capability/ssgx/software-solutions/accounting.html}{Princeton Financial Systems}底层技术的维护和开发;
  % 		\item  将旧系统中C语言写的部分重构成C++模块,并优化旧系统的性能。
  % 	\end{itemize}
  % }
  % {C/C++编程, 性能优化}

\end{experiences}
