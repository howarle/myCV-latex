% Awesome CV LaTeX Template
%
% This template has been downloaded from:
% https://github.com/huajh/huajh-awesome-latex-cv
%
% Author:
% Junhao Hua

% “STAR”法则
% 情境(Stuation):写出你的工作背景
% 任务(Task):我负责做什么
% 行动(Action):我做了什么
% 结果(Result):我的工作取得了什么样的结果

\sectionTitle{项目经历}{\faCode}

\begin{experiences}
  \experiencenew
  {2022.09 -- 2023.01} {CCF BDCI 2022 基于文心CV大模型的智慧城市视觉多任务识别,赛题项目}
  {
    赛题大意为神经架构性能预测,即给出多个不同的神经网络,选手需要根据神经网络的结构,预测这些网络的性能相对排名;训练集大小为500,预测集大小为90k。
    \begin{itemize}
      \item  作为组长,负责前期调研、Baseline分析、任务分配;
      \item  搭建项目整体框架,统一开发实验流程,实现测试自动化,方便团队进行模型的选择及调参;
      \item  负责后续决赛材料筹备,决赛答辩;
    \end{itemize}
    在赛题中,我创新性的将传统算法与机器学习模型相结合,为团队取得了巨大优势,在Leaderboard A/B上均遥遥领先第二名,并顺利通过答辩获得冠军。 \\
    项目地址:\link{https://aistudio.baidu.com/aistudio/projectdetail/5035322}{【2022 CCF BDCI 基于文心CV大模型的智慧城市视觉多任务识别】第1名方案\faExternalLink}
  }
  {Python, TensorFlow, 数据挖掘, 机器学习, 神经架构搜索}
  \emptySeparator

  \experiencenew
  {2022.01 -- 2022.04} {华为openGauss-MySQL在线迁移工具支持迁移对象合作项目,校企合作项目}
  {
    基于chameleon开发MySQL到openGauss的全量对象迁移工具,涉及到sql语言翻译工作。
    使用阿里的druid来对MySQL的语句进行解析建立AST,并重新遍历AST翻译出其对应的gpglsql语句。
    \begin{itemize}
      \item  负责前期Mysql与og的差异分析、sql翻译工具的调研测试、druid的流程探索;
      \item  项目中期Druid重载开发工作,开发进度统筹;
      \item  撰写测试方案、测试用例;
    \end{itemize}
  }
  {JAVA, Python, SQL, openGauss数据库, Mysql数据库}
  \emptySeparator

  \experiencenew
  {2021.10 -- 2022.01} {华为openGauss兼容MySQL时间数据类型及函数合作项目,校企合作项目}
  {
    以插件的形式,使得openGauss兼容MySQL时间数据类型及时间处理函数。
    \begin{itemize}
      \item  主要负责MySQL源码调研、兼容性方案设计、边界范围的划定;
      \item  完成完成时间类型的存储方案设计、实现、UT,及部分周边功能函数的支持;
      \item  建立自动化测试的流程,提高开发、测试的效率;
    \end{itemize}
    此项目还被导师作为数据库课程实践,因此我还作为助教,参与课程实践的答疑及评审工作。
  }
  {C/C++, SQL, openGauss数据库, Mysql数据库}

  % \experiencenew
  % {2022.04 -- 2022.09} {华为openGauss-Mysql兼容性时间类型相关函数开发合作项目,校企合作项目}
  % {
  %   基于《华为openGauss兼容MySQL时间数据类型开发合作项目》,继续在openGuass数据库中对MySQL时间类型函数进行兼容开发。
  %   \begin{itemize}
  %     \item  主要负责MySQL源码调研、兼容性方案设计、边界范围的划定;
  %     \item  项目中期的代码开发及文档撰写;
  %   \end{itemize}
  %   此项目还被导师作为数据库课程实践,因此我还作为助教,参与课程实践的答疑及评审工作。
  % }
  % {C/C++, SQL, openGauss数据库, Mysql数据库}
  % \emptySeparator

  % \experiencenew
  % {2021.10 -- 2022.01} {华为openGauss兼容MySQL时间数据类型开发合作项目,校企合作项目}
  % {
  %   对华为openGauss数据库进行功能扩展,对openGauss兼容MySQL时间数据类型作兼容性开发。
  %   \begin{itemize}
  %     \item  作为实验室第一个og项目,在没有开发文档、资料极少的情况下,仅凭借用户手册及gdb调试,定位功能模块代码,并推断数据类型注册流程、处理流程。
  %     \item  完成完成year,year2类型的方案设计、核心代码的实现及部分周边功能函数的支持;
  %     \item  添加fastcheck测试用例、撰写og升级脚本;
  %   \end{itemize}
  %   由于og的兼容性开发一直在持续进行,后续合入代码可能引起兼容性问题,我还负责此项目的后续问题的修复;
  % }
  % {C/C++, SQL, openGauss数据库, Mysql数据库}

  % \emptySeparator
  % \experience
  % {2012年11月} {C/C++ 工程师实习生}{研发部分}{道富信息科技(浙江)有限公司, 杭州}
  % {2012年7月}    {
  % 	\begin{itemize}
  % 		\item  负责\link{http://www.statestreet.com/solutions/by-capability/ssgx/software-solutions/accounting.html}{Princeton Financial Systems}底层技术的维护和开发;
  % 		\item  将旧系统中C语言写的部分重构成C++模块,并优化旧系统的性能。
  % 	\end{itemize}
  % }
  % {C/C++编程, 性能优化}

  % \emptySeparator
  % \experience
  % {2012年7月} {项目组成员}{智能系统研究所}{计算机学院,浙江工业大学}
  % {2011年5月 }    {
	% 			  	\begin{itemize}
	% 			  		\item  2011年10月- 2012年5月, 研究\emph{基于空间复杂网络的交通路由算法};
	% 			  		主要思想是通过\emph{潜在度量空间}建立空间网络模型,设计基于全局和局部信息的路由策略及其导航性能。
	% 			  		\item  2011年5月 - 9月,  以Microsoft Kinect SDK (C\#)为开发工具编程实现\emph{PPT体感操控系统设计}。
	% 			  	\end{itemize}
	% 			  }
	% 			  {complex networks, kinect, C\#}



  % \emptySeparator
  % \experience
  % {2011年12月} {\emph{软件开发}}{浙江工业大学}{本科小项目}
  % {2011年10月}    {
	% 			  	\begin{itemize}
	% 			  		\item Oct-Dec 2011, \emph{竞赛作品展示平台} | JavaBeans+Servlet+Jsp框架| \emph{队长}.
	% 			  		独立设计并实现了JDBC的轻量级对象关系封装,并用于服务端开发, 淘宝UED评审获第二名。
	% 					  		\faGithub: \link{https://github.com/huajh/showplatform}{github.com/huajh/showplatform}
	% 			  		\item Nov 2011, \emph{Unix文件系统} | C/C++ | \emph{独立开发}.
	% 					  		实现系统的格式化、安装、加载,用户组管理,打开文件管理,内存分配,文件读写,以及基本的shell命令,
	% 					  		包括目录文件的添加、删除、重命名、拷贝等。
	% 					  		\faGithub: \link{https://github.com/huajh/unix_file_sys}{github.com/huajh/unix\_file\_sys}
	% 			  	\end{itemize}
	% 			  }
	% 			  {软件开发, 数据库, JAVA, Unix, Sql Server}


\end{experiences}
